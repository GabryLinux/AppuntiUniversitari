\subsection{Proprietà dei Segnali}
\subsubsection{Durata}
La durata di un segnale è, considerando un segnale nel tempo, la differenza tra il primo istante in cui il segnale non è nullo e l'ultimo istante.

\subsubsection{Area}
Si dice \textit{Area di un Segnale} $s(t)$ l'area sottesa dallo stesso segnale, ossia:
\begin{equation}
    \int_{-\infty}^{+\infty} s(t)dt
\end{equation}

\subsubsection{Valor Medio (o Media Temporale)}
Il valor medio di un segnale $s(t)$ non è altro che quel valore $\tilde{s}$ tale che una funzione costante $s'(t) = \tilde{s}$ ha la stessa area di $s(t)$, ossia:
\begin{equation}
    \tilde{s} = \lim_{T \rightarrow +\infty} \frac{1}{2T} \int_{-T}^{+T}s(t)dt
\end{equation}

\subsubsection{Energia}
Sebbene non stiamo parlando propriamente di lavoro e concetti fisici relativi, dobbiamo dire che un segnale è sempre associato ad una certa energia che il segnale stesso trasporta.
Dunque \textit{l'energia di un segnale $(s(t))$} è:
\begin{equation}
    E_s = \int_{-\infty}^{+\infty} |s(t)|^2 dt
\end{equation}

\subsubsection{Potenza Istantanea}
Per il discorso precedente, possiamo anche definire la Potenza Istantanea di un segnale (cioè la potenza in un istante del Segnale) come
\begin{equation}
    P[s(t)] = \begin{cases}
        s(t_0)\overline{s(t_0)} & \mbox{ se } s(t_0) \in \mathbb{C} \\
        s(t_0)^2 & \mbox{ se } s(t_0) \in \mathbb{R}
    \end{cases}
\end{equation}

\subsubsection{Potenza Media}
La potenza media possiamo, invece, vederla come il valore medio dell'energia, ossia:
\begin{equation}
    P_s = \lim_{T \rightarrow +\infty} \frac{1}{2T} \int_{-T}^{+T}|s(t)|^2 dt
\end{equation}

\subsubsection{Segnale Energia e Potenza} \label{def: sep}
Innanzitutto possiamo notare come sia Potenza Media che Energia siano non negativi per costruzione;
inoltre tra Potenza Media e Energia di un segnale è possibile vedere una correlazione: laddove l'energia del segnale è finita, allora la potenza è necessariamente nulla;
laddove invece la potenza media è maggiore di 0, l'energia è infinita.

In base a questo concetto è possibile definire:
\begin{itemize}
    \item \textbf{Segnale Energia} un segnale $s(t)$ se e solo se $0 < E_s < \infty$ e allora $P_s = 0$
    \item \textbf{Segnale Potenza} un segnale $s(t)$ se e solo se $0 < P_s < \infty$ e allora $E_s \rightarrow +\infty$
\end{itemize}

\newpage

\section{Segnali Discreti}
A differenza dei segnali continui, i segnali discreti sono funzioni con Dominio discreto, solitamente rappresentate nella seguente maniera:
\begin{equation*}
    y = f(n), \mbox{  } n \in Z
\end{equation*}
Enunciamo ora tutte le proprietà, in maniera speculare, che abbiamo già descritto per i segnali continui:
\subsection{Operazione sui Segnali Discreti}
\subsubsection{Traslazione}
Dato un segnale $y = f(n)$ definiamo $y' = f(n - n_0)$ traslazione in avanti di $n_0$; definiamo invece $y'' = f(n + n_0)$ traslazione indietro di $n_0$.
\subsubsection{Decimazione/UpSampling}
Dato $y=f(n)$ il segnale con $n \in \mathbb{Z}$, il segnale decimato è:
\begin{equation}
    y = f(an) \mbox{ con } a \in Z \mbox{ e } |a| \geq 1 
\end{equation}
Questa operazione è detta \textit{decimazione} dal momento che è come se prelevassi selettivamente i valori ogni $a$ campioni del segnale di partenza

\subsubsection{Interpolazione/DownSampling}
Dato $y=f(n)$ il segnale con $n \in \mathbb{Z}$, il segnale interpolato è:
\begin{equation}
    y = f\left(\frac{n}{a}\right) \mbox{ con } a \in Z \mbox{ e } |a| \geq 1 
\end{equation}
In parole povere, questa operazione non fa altro che distanziare ogni campione di $a$ intervalli.

\newpage

\subsection{Segnali Notevoli}
\subsubsection{Rettangolo Discreto}
Viene definito come:
\begin{equation}
    rect\left(\frac{n}{D}\right) = \begin{cases}
        1 & \mbox{ se } |\frac{n}{D}| \leq \frac{1}{2}\\
        0 & \mbox{ altrimenti}
    \end{cases}
\end{equation}
\begin{center}
    \begin{tikzpicture}
        % Assi
        \draw[->] (-4,0) -- (4,0) node[below] {$n$}; % asse del tempo discreto
        \draw[->] (0,-2) -- (0,2) node[left] {$rect\left(\frac{n}{D}\right)$}; % asse dell'ampiezza
    
        
        \draw[dashed] (-2,0) -- (-2,1);
        \filldraw[blue] (-2,1) circle (2pt);
        
        \draw[dashed] (-1,0) -- (-1,1);
        \filldraw[blue] (-1,1) circle (2pt);
    
        \draw[dashed] (0,0) -- (0,1);
        \filldraw[blue] (0,1) circle (2pt);
    
        \draw[dashed] (1,0) -- (1,1);
        \filldraw[blue] (1,1) circle (2pt);
    
        \draw[dashed] (2,0) -- (2,1);
        \filldraw[blue] (2,1) circle (2pt);
    
        \filldraw[blue] (-3,0) circle (2pt);
        \filldraw[blue] (3,0) circle (2pt);
        \filldraw[blue] (-4,0) circle (2pt);
        \filldraw[blue] (4,0) circle (2pt);
    
    
    
        % Etichettatura degli intervalli di campionamento
        \draw (-2,0) node[below] {$-\frac{D}{2}$};
        \draw (2,0) node[below] {$\frac{D}{2}$};
        \draw (0,1) node[left] {$1$};
    \end{tikzpicture}
\end{center}

\subsubsection{Gradino Unitario}
Anche in questo caso, la formulazione è identica al Gradino Unitario continuo, ossia:
\begin{equation}
    u(t) = \begin{cases}
        1 & \mbox{ se } n \geq 0\\
        0 & \mbox{ se } n < 0
    \end{cases} 
\end{equation}

\subsubsection{Impulso Discreto}
Questo segnale è il parente "discreto" della \textit{Delta di Dirac} (\ref{eq:delta}) ma è più semplice da introdurre, dal momento che la sua formula è:
\begin{equation}
    \delta(n) = \begin{cases}
        1 & \mbox{ se } n = 0\\
        0 & \mbox{ se } n \neq 0
    \end{cases}
\end{equation}
Possiamo dunque notare che, a differenza della delta, l'impulso discreto è una vera e propria funzione.

\subsubsection{Proprietà dell'Impulso Discreto}
\begin{itemize}
    \item \textbf{Area Unitaria}: è abbastanza ovvia
    \begin{equation}
        \sum_{n=-\infty}^{+\infty} \delta(n) = 1
    \end{equation}

    \item \textbf{Prodotto Scalare con $\delta(n)$} 
    \begin{equation}
        \langle f,\delta \rangle = \sum_{n= -\infty}^{\infty} f(n)\delta(n) = f(0) 
    \end{equation}

    \item \textbf{Prodotto con $\delta(n)$}:
    \begin{equation}
        f(n)\delta(n - n_0) = f(n_0)\delta(n - n_0)
    \end{equation}

    \item \textbf{Integrazione Discreta}:
    \begin{equation}
        \sum_{i = -\infty}^{\infty} \delta(i) = \begin{cases}
            1 & \mbox{ se } n \geq 0\\
            0 & \mbox{ se } n < 0
        \end{cases} = u(n)
    \end{equation}
\end{itemize}

\subsubsection{Segnali Periodici Discreti}
Dato $s(n)$ un segnale con $n \in \mathbb{Z}$ è periodico se e solo se
\begin{equation}
    s(n) = s(n + kN), \mbox{ con periodo }N, \forall n,k \in \mathbb{Z}
\end{equation}
Se $N$ è il periodo di $s(n)$, allora la frequenza sarà $\frac{1}{N}$; Dal momento che $|N| \geq 1$ sempre (non ha senso avere periodo nullo), allora la frequenza è sempre frazionaria.
Si può dimostrare che solo un segnale con frequenza razionale può essere periodico, mentre se è irrazionale non lo potrà mai essere.

\subsubsection{Fasore Discreto}
La funzione è molto simile al fasore continuo:
\begin{equation}
    s(n) = e^{2\pi j f_0 n}
\end{equation}
Dove $f_0$ dev'essere razionale altrimenti la funzione non sarà periodica.