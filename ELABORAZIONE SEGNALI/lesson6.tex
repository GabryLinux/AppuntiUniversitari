\subsection{Proprietà dei Segnali Discreti}
\subsubsection{Durata}
La durata di un segnale discreto è la somma delle "stecche" non nulle di un grafico, o meglio, la lunghezza del supporto di $s(t)$
\begin{equation}
    D = n_2 - n_1 + 1
\end{equation}
Dove $n_2,n_1$ sono gli estremi non nulli del segnale.

\subsubsection{Area}
Dato $s(n), n \in \mathbb{Z}$
\begin{equation}
    A = \sum_{n = -\infty}^{+\infty} s(n)
\end{equation}

\subsubsection{Valor Medio}
Il valor medio è un $\tilde{s}$ tale che la funzione costante $s(t)' = \tilde{s}$ ha la stessa area di $s(t)$
\begin{equation}
    \tilde{s} = \lim_{N \rightarrow +\infty} \frac{1}{2N + 1} \sum_{n=-N}^{+N} s(n)
\end{equation}

\subsubsection{Potenza Istantanea}
Non è altro che il modulo del segnale in un determinato istante $n$, ossia:
\begin{equation}
    P_s(n) = \begin{cases}
        s(n)\overline{s(n)} \mbox{ se } s(n) \in \mathbb{C} \\
        s(n)^2 \mbox{ altrimenti}
    \end{cases}
\end{equation}

\subsubsection{Energia}
è l'area della Potenza Istantanea, ossia:
\begin{equation}
    E_s = \sum_{n=-N}^{+N} |s(n)|^2
\end{equation}

\subsubsection{Potenza Media}
è il valor medio della Potenza Istantanea, ossia:
\begin{equation}
    P_s =  \lim_{N \rightarrow +\infty} \frac{1}{2N + 1} \sum_{n=-N}^{+N} |s(n)|^2
\end{equation}

\subsubsection{Segnali Potenza ed Energia}
Anche in questo caso possiamo fare la distinzione tra i segnali potenza e i segnali energia, la definizione è la stessa presentata 
alla sezione \eqref{def: sep}

\newpage

\section{Sistemi}
In forma generale, un sistema descrive una relazione/processo di Causa-Effetto tra un INPUT ed un OUTPUT. In particolare nel corso 
ci focalizzeremo sui sitemi di Elaborazione dei Segnali.\\
Un \textbf{Sistema di Elaborazione dei Segnali} può essere vista come una relazione o legge di trasformazione di un segnale in un altro,
dunque, formalmente:\\
Dato un sistema $S[\cdot]$, possiamo dire che:
\begin{equation}
    y(b) = S[x(t)]
\end{equation}
Dove $y(b)$è il segnale in output mentre $x(t)$ è il segnale in input.
In base alla continuità dei segnali e dei domini possiamo distinguere i sistemi in: \textbf{Continui, Discreti e Misti}

\subsection{Sistemi Continui}
Dato un sistema $y(b) = S[x(a)]$, con $b \in B, a \in A$ questo si dice continuo se e solo se:
\begin{itemize}
    \item $A,B$ sono insiemi continui
    \item $y(b),x(a)$ sono funzioni continue
\end{itemize}
Una classe importante di questi sistemi sono i \textbf{Sistemi Tempo-Continui} in cui la variabile è il tempo.

\paragraph{Esempi. }
\begin{itemize}
    \item \textbf{Ritardatori}
    \begin{equation}
        y(t) = S[x(t)] = x(t - t_0)
    \end{equation}
    Questo sistema non fa altro che ritardare il segnale $x(t)$ di una quantità di $t_0$ secondi.

    \item \textbf{Quantizzatore}
    \begin{equation}
        y(t) = \mbox{ROUND}(x(t))
    \end{equation}
    Questo sistema arrotonda ogni valore di $x(t)$ al suo intero più vicino.

    \item \textbf{Integratore}
    \begin{equation}
        y(t) = S[x(t)] = \int_{t - T}^{t} x(\tau)d\tau
    \end{equation}
    Questo sistema associa ad ogni istante $t$ l'area compresa tra $t - T$ e $t$
\end{itemize}

\subsection{Proprietà dei Sistemi Tempo-Continui}
\subsubsection{Non Dispersività}
Un sistema $S[\cdot]$ è non dispersivo se e solo se il sistema dipende solo da $t$ e dal valore attuale di $x(t)$, ossia:
\begin{equation}
    y(t) = S[t; x(t)]
\end{equation}
possiamo dire che questi sistemi sono senza memoria perchè non considerano per nulla il passato.
\paragraph{Esempio:}
\textbf{Amplificatore Ideale}
\begin{equation}\label{eq: AmpId}
    y(t) = A \cdot x(t)
\end{equation}

\subsubsection{Causalità}
Un sistema $S[\cdot]$ è causale se e solo se:
\begin{equation}
    y(t) = S[t; x(\tau) \mbox{ dove } \tau \leq T]
\end{equation}
Questi sistemi associano ad ogni istante t un valore dipendente non soltanto dal tempo attuale e dal valore $x(t)$, ma 
anche della storia di $x(t)$. Bene o male tutti i sistemi reali sono causali, perchè dipendono anche dagli istanti passati di un segnale.
Esistono (ma solo in teoria) i segnali \textbf{Anticausali}, in cui il sistema dipende dall'istante attuale e quelli futuri (e quindi il futuro causerebbe il presente, impossibile nella realtà)

\subsubsection{Stabilità BIBO (Bounded Input, Bounded Output)}
Un sistema è \textbf{Stabile} se e solo se per ogni input limitato, l'uscita è sempre limitata, ossia:\\
Dato $S[\cdot]$ dove $y(t) = S[x(t)]$ con $|x(t)| \leq K_x < +\infty$  $\forall t \in \mathbb{R}$ allora:
\begin{equation}
    |y(t)| \leq K_y < +\infty \mbox{  } \forall t \in \mathbb{R}
\end{equation}
Dove $K_x,K_y$ sono rispettivamente il limite superiore/inferiore di $x(t),y(t)$
\paragraph{Esempi:} L'Amplificatore Ideale \eqref{eq: AmpId}.