\section{Segnali}
In termini totalmente astratti, un segnale è un veicolo di informazione. 
L'informazione, a sua volta, possiamo definirla come tutto ciò che aggiunge conoscenza.
Calato nel contesto del corso, un segnale può essere visto come una funzione
\begin{equation*}
    \begin{cases}
        y = f(x)\\
        f : A \longrightarrow B
    \end{cases}
\end{equation*}
Dove $x \in A$ è la variabile indipendente, $y \in B$ la variabile dipendente con $A$ e $B$ insiemi qualsiasi.
\subsection{Classificazione dei segnali}
\subsubsection{Rispetto alla dimensionalità}
Un segnale può essere classificato rispetto alla dimensionalità del dominio o del codominio.
\paragraph{Dominio.} Un segnale rispetto alla dimensionalità del dominio può essere:
\begin{itemize}
    \item \textbf{Monodimensionale} se $A \subseteq \mathbb{R}$
    \item \textbf{n-Dimensionale} se $A \subseteq \mathbb{R}^n$ con $n > 1$
\end{itemize}
\paragraph{Codominio.} Un segnale rispetto alla dimensionalità del codominio può essere:
\begin{itemize}
    \item \textbf{Scalare} se $B \subseteq \mathbb{R}$
    \item \textbf{Vettoriale} se $B \subseteq \mathbb{R}^n$ con $n > 1$
\end{itemize}

\subsubsection{Rispetto alla Continuità}
Un segnale può essere classificato rispetto alla continuità di dominio o codominio.
\paragraph{Dominio.} Un segnale rispetto alla continuità del dominio può essere:
\begin{itemize}
    \item \textbf{Continuo} se $A \subseteq \mathbb{R}^n$, cioè se $A$ coincide o è sottoinsieme di un insieme denso, continuo.
    \item \textbf{Discreto} se $A$ coincide o è sottinsieme di un insieme discreto (come $\mathbb{Z}$ o $\mathbb{N}$)
\end{itemize}
\paragraph{Codominio.} Un segnale rispetto alla continuità del codominio può essere:
\begin{itemize}
    \item \textbf{Ad Ampiezze Continue} se $B \subseteq \mathbb{R}^n$, cioè se $B$ coincide o è sottoinsieme di un insieme denso, continuo.
    \item \textbf{Ad Ampiezze Discrete} se $B$ coincide o è sottinsieme di un insieme discreto (come $\mathbb{Z}$ o $\mathbb{N}$)
\end{itemize}
Nel caso in cui la variabile dipendente sia il tempo, i segnali vengono detti \textbf{tempo-contuinui} o \textbf{tempo-discreti}.

\paragraph{Esempi.}
Il \textit{suono} può essere visto come la pressione dell'aria in funzione del tempo, ossia come:
\begin{equation*}
    p = f(t)
\end{equation*}
Dove $p \in \mathbb{R}$, $t \in \mathbb{R}$. Dunque il suono è un segnale \textbf{monodimensionale} e \textbf{scalare}, \textbf{continuo} e \textbf{ad ampiezze continue}\\

Un'\textit{immagine in bianco e nero} può essere vista come una funzione che associa ad ogni punto del piano la \textit{Luminanza}:
\begin{equation*}
    L = f(x,y)
\end{equation*}
Dove $L \in \mathbb{R}$ e $\langle x,y \rangle \in \mathbb{R}^2 $. Dunque un'immagine a scala di grigi è un segnale \textbf{bidimensionale} e \textbf{scalare}, \textbf{continuo} e \textbf{ad ampiezze continue}\\

Un'\textit{immagine a colori} può essere vista come una funzione che associa ad ogni punto del piano una tripletta di valori (interpretabili in base allo spazio colore scelto). Consideriamo lo spazio RGB:
\begin{equation*}
    \langle R,G,B \rangle = f(x,y)
\end{equation*}
Dove $\langle R,G,B \rangle \in \mathbb{R}^3$ e $\langle x,y \rangle \in \mathbb{R}^2 $. Dunque un'immagine a colori è un segnale \textbf{bidimensionale} e \textbf{vettoriale}, \textbf{continuo} e \textbf{ad ampiezze continue}

\subsection{Tipi di segnali}
In base alle caratteristiche di un segnale, possiamo definire alcuni tipi di segnali:
\begin{table}[h]
    \centering
    \begin{tabular}{|c|c|c|}
        \hline
        Segnale & Dominio Continuo & Dominio Discreto \\ \hline
        Codominio Continuo &  Analogico &  Campionato \\ \hline
        Codominio Discreto &  Quantizzato &  Digitale \\ \hline
    \end{tabular}
    \end{table}