\subsubsection{Proprietà}
\begin{itemize}
    \item \textbf{Commutativa:}dati 2 segnali $f,g$
    \begin{equation} \label{prop: convComm}
        f(t) \ast g(t) = g(t) \ast f(t)
    \end{equation}
    Tale proprietà è facile da dimostrare:\\
    Consideriamo la convoluzione $f(t) \ast g(t) = \int_{-\infty}^{+\infty} f(\tau)g(t - \tau) d\tau$. Effettuiamo ora un cambio di variabile:
    \begin{equation*}
        \begin{cases}
            x = t - \tau\\
            \tau = t - x\\
            dx = -d\tau
        \end{cases}
    \end{equation*}
    Sostituiamo ora le $\tau$ per avere un integrale in $x$ (NB: gli estremi di integrazione si invertono dal momento che $x$ è un "ribaltamento" di $\tau$):
    \begin{align*}
        \int_{-\infty}^{+\infty} f(\tau)g(t - \tau) d\tau= \\
        \int_{+\infty}^{-\infty} f(t - x)g(x) (-dx) = \\
        \int_{-\infty}^{+\infty} g(x)f(t - x) dx = \\
        g(t) \ast f(t)
    \end{align*}
    Alla fine ho risolto il $-dx$ semplicemente invertendo ancora gli estremi di integrazione. Il significato di questa proprietà è importante:
    se è vero che è possibile studiare un sistema LTI rispetto alla sua risposta all'impulso, è vero anche il contrario, è possibile studiare un sistema rispetto ad un segnale $x(t)$
    \item \textbf{Associativa}: dati 3 segnali $f,g,h$ allora:
    \begin{equation}
        f(t) \ast [g(t) \ast h(t)] = [f(t) \ast g(t)] \ast h(t)
    \end{equation}

    \item \textbf{Distributiva (rispetto alla somma)}: dati 3 segnali $f,g,h$: \label{prop: ConvDistr}
    \begin{equation}
        f(t) \ast [g(t) + h(t)] = f(t) \ast g(t) + f(t) \ast h(t) 
    \end{equation}

    \item \textbf{Durata della Convoluzione}: \label{prop: ConvDurata} dati 2 segnali $f,g$ con durata rispettivamente $D_1,D_2$ allora la durata di $f(t) \ast g(t)$ è $D_1 + D_2$
    
    \item \textbf{Convoluzione con $\delta(t)$}: \label{prop: ConvDelta} dato un segnale $f(t)$ allora:
    \begin{equation}
        f(t) \ast \delta(t - t_0) = f(t - t_0)
    \end{equation}
    Possiamo infatti dire che $\delta(t)$ è l'operatore neutro dell'operazione di convoluzione, questo perchè il delta è una funzione che avviene tutta in un istante e quindi non altera la funzione di partenza; una sua eventuale traslazione, trasla la funzione $f(t)$.
    La dimostrazione è banale perchè si basa sulla proprietà del campionamento della delta \eqref{eq: camp}
\end{itemize}


\newpage

\subsection{Proprietà dei Sistemi LTI Continui}
\subsubsection{Causalità} \label{prop: causalita}
Dato un sistema LTI $S[\cdot] : y(t) = S[x(t)]$, $S$ è causale se $y(t)$ accade DOPO il segnale $x(t)$, questo per il concetto stesso di causalità:
$y(t)$ è la risposta all'impulso di $x(t)$ quindi DEVE accadere dopo, in particolare si dice che:\\
\begin{equation}
    \mbox{Un sistema è causale} \Longleftrightarrow h(t) = 0 \mbox{  } \forall t < 0
\end{equation}
Proprio per questo motivo, dal momento che $y(t)$ può essere descritto come risposta all'impulso, ossia come convoluzione tra $x(t)$ e $h(t)$, se 
il sistema è causale allora possiamo ridefinire la $y(t)$ come integrale dall'infinito passato fino all'istante $t$ presente, ossia:
\begin{equation}
    y(t) = \int_{-\infty}^{t} x(\tau) h(t - \tau) d\tau
\end{equation}
Possiamo anche, per la proprietà commutativa \eqref{prop: convComm}, riscrivere la $y(t)$ come integrale tra la risposta $h(t)$ e il segnale $x(t)$ ribaltato e traslato
di tutti i valori di $\tau$ da 0 fino all'infinito futuro
\begin{equation}
    y(t) = \int_{0}^{+\infty} h(\tau)x(t - \tau) d\tau
\end{equation}

\subsubsection{Stabilità BIBO} \label{prop: BIBO}
Dato un sistema LTI $S[\cdot] : y(t) = S[x(t)]$, $S[\cdot]$ è BIBO se e solo se:
\begin{equation}
    \forall x(t) : |x(t)| < M < \infty \rightarrow |y(t)| < N < \infty
\end{equation}
In particolare possiamo osservare che:
\begin{equation*}
    |y(t)| = |S[x(t)]| = \left|\int_{-\infty}^{+\infty} h(\tau) x(t - \tau) d\tau \right|
\end{equation*}
Sappiamo inoltre che, data una qualsiasi $f(x)$ vale questa disuguaglianza:
\begin{equation*}
    \left|\int_{-\infty}^{+\infty} f(x)dx \right| \leq \int_{-\infty}^{+\infty} |f(x)|dx
\end{equation*}
Dunque
\begin{equation*}
    y(t) \leq \int_{-\infty}^{+\infty} |h(\tau) x(t - \tau)|d\tau
\end{equation*}
Per la proprietà del valore assoluto $|ab| = |a||b|$ allora:
\begin{equation*}
    \int_{-\infty}^{+\infty} |h(\tau) x(t - \tau)|d\tau = \int_{-\infty}^{+\infty} |h(\tau)| |x(t - \tau)|d\tau
\end{equation*}
Ma $x(t - \tau) \leq M$ perchè abbiamo ipotizzato che il segnale in ingresso sia limitato, dunque
\begin{equation*}
    \int_{-\infty}^{+\infty} |h(\tau)| |x(t - \tau)|d\tau \leq M\int_{-\infty}^{+\infty} |h(\tau)|d\tau
\end{equation*}
In definitiva:
\begin{equation*}
    y(t) \leq M\int_{-\infty}^{+\infty} |h(\tau)|d\tau
\end{equation*}
Possiamo dunque affermare che condizione necessaria e sufficiente affinchè un sistema sia stabile BIBO è che $\int_{-\infty}^{+\infty} |h(\tau)|d\tau$ converga, altrimenti $y(t)$ è illimitato.\\
\paragraph{Teorema.} Se la risposta all'impulso del Sistema non è assolutamente convergente allora non è BIBO Stabile
\paragraph{Ipotesi. } Ipotizziamo che:
\begin{enumerate}
    \item Il sistema sia BIBO Stabile, $|y(t)| < N < \infty$
    \item $\int_{-\infty}^{+\infty} |h(\tau)| \tau = + \infty$
    \item consideriamo una $x(t)$ limitata, ossia $x(t) = \mbox{sign}[h(-t)]$ 
\end{enumerate}
\paragraph{Dimostrazione. } Possiamo dire che per $t = 0$ la risposta all'impulso è:
\begin{align*}
    &y(t) = \int_{-\infty}^{+\infty} h(\tau) x(0 -\tau) d\tau =  \int_{-\infty}^{+\infty} h(\tau)\mbox{sign}[h(-\tau)] d\tau
\end{align*}
per $t = 0$, $\mbox{sign}[h(-\tau)] = 1$ dunque:
\begin{equation*}
    \int_{-\infty}^{+\infty} h(\tau)\mbox{sign}[h(-\tau)] d\tau = \int_{-\infty}^{+\infty} h(\tau) d\tau
\end{equation*}
Inoltre
\begin{align*}
    \int_{-\infty}^{+\infty} h(\tau) d\tau &\leq \int_{-\infty}^{+\infty} |h(\tau)| d\tau = |y(t)| \\
    |y(t)| &= \int_{-\infty}^{+\infty} |h(\tau)| d\tau = \infty
\end{align*}
ASSURDO, allora $y(t)$ è instabile.\\Possiamo dunque affermare che:
Un sistema è BIBO se e solo se:
\begin{equation}
    \int_{-\infty}^{+\infty} |h(\tau)| d\tau < \infty
\end{equation}

\subsubsection{Risposta in frequenza}
Consideriamo la risposta del sistema rispetto al fasore di frequenza $f$:
\begin{equation*}
    x_f(t) = e^{j2\pi ft}
\end{equation*}
Dal momento che il sistema è LTI allora:
\begin{align*}
    y(t) = h(t) \ast x_f(t) = \\
    = \int_{-\infty}^{+\infty} h(\tau)e^{j2\pi f(t - \tau)} d\tau =\\
    = \int_{-\infty}^{+\infty} h(\tau)e^{j2\pi ft}e^{-j2\pi\tau} d\tau =\\
    = e^{j2\pi ft} \int_{-\infty}^{+\infty} h(\tau) e^{-j2\pi\tau} d\tau 
\end{align*}
\begin{highlightedeq}
    Possiamo vedere $\int_{-\infty}^{+\infty} h(\tau) e^{-j2\pi\tau} d\tau$ come una sorta di convoluzione in 0, che noi chiamiamo $H(f)$ 
    e definiamo \textbf{Risposta in Frequenza}. Possiamo riscrivere l'equazione come:
    \begin{equation}
        y(t) = H(f)x_f(t)
    \end{equation}
\end{highlightedeq}
Da $H(f)$ possiamo definire:
\begin{itemize}
    \item \textbf{Risposta in Ampiezza}: $|H(f)|$
    \item \textbf{Risposta in Fase}: $\angle H(f)$
\end{itemize}
Questa proprietà dei LTI è molto importante poichè ci dice che la risposta in frequenza di un qualsiasi segnale $x(t)$ è un altro segnale \textbf{alla stessa frequenza}, magari traslato o amplificato,
cioè la risposta di un fasore è ancora un fasore di frequenza $f$ ma moltiplicato per un valore complesso $H(f)$



\subsection{Proprietà Sistemi Discreti}
Un sistema $S[\cdot]:y(t) = S[x(t)]$ è detto discreto se e solo se $x,y$ sono segnali discreti; possiamo enunciare per tali sistemi le stesse proprietà dei sistemi LTI continui:
\paragraph{Causalità.}
Ha la medesima formulazione dei sistemi LTI Continui \eqref{prop: causalita} 

\paragraph{Stabilità BIBO.}
Ha la medesima formulazione dei sistemi LTI Continui \eqref{prop: BIBO} 

\paragraph{Linearità.}
Ha la medesima formulazione dei sistemi LTI Continui \eqref{prop: linearita} 

\paragraph{Tempo-Invarianza.} 
La medesima formulazione dei sistemi LTI Continui \eqref{prop: TI} 