\subsubsection{Potenza Del Disturbo}
Una volta individuata la forma e l'entità del disturbo, possiamo calcolare la potenza di questo segnale:
\begin{equation*}
    P_Q = \int_{-\infty}^{+\infty} e^2 f_{eq}(e)de
\end{equation*}
Abbiamo visto prima che, se gli intervalli $\Delta$ sono sufficientemente piccoli, possiamo assumere che $f_{eq}(e)$ sia una
funzione di densità di probabilità uniforme, che ha questa forma:
\begin{equation}
    f_{eq}(e) = \begin{cases}
        \frac{1}{\Delta} &\text{ se } -\frac{1}{\Delta} \leq e \leq \frac{1}{\Delta}\\
        0                &\text{ altrimenti}
    \end{cases}
\end{equation}
Allora possiamo riscrivere l'equazione precedente:
\begin{align*}
    P_Q &= \int_{-\frac{\Delta}{2}}^{+\frac{\Delta}{2}} \frac{1}{\Delta} e^2 de =\\
        &= \frac{1}{\Delta} \int_{-\frac{\Delta}{2}}^{+\frac{\Delta}{2}} e^2 de =\\
        &= \frac{1}{\Delta} \left( \frac{\Delta^3}{24} +  \frac{\Delta^3}{24}\right) =\\
        &= \frac{\Delta^2}{12}
\end{align*}
Quello che abbiamo trovato è la \textbf{Potenza Media dell'errore di Quantizzazione}. 

\paragraph{Signal Noise Ratio.}La potenza Media dell'errore, tuttavia,
ci dice ben poco poichè tale valore dev'essere sempre confrontato con la potenza del segnale disturbato,
per questo motivo ha più senso parlare del rapporto, anche detto \textbf{SIgnal Noise Ratio (SNR)}:
\begin{equation}
    \text{SNR} = \frac{P_\text{signal}}{P_\text{noise}}
\end{equation}
I possibili $P_\text{signal}$ che possiamo scegliere per calcolare quel rapporto sono la \textbf{Potenza di Picco} o la \textbf{Potenza Media}

\paragraph{Potenza Di Picco.}Sappiamo che il valore $x$ del segnale da quantizzare è sempre limitato, ossia:
\begin{equation*}
    V_{MIN} \leq x < V_{MAX}
\end{equation*}
Dove $\{V_{MIN}; V_{MAX}\}$ è detta \textbf{Escursione del Segnale}. Considereremo, per semplicità, \textbf{segnali bipolari}
ossia segnali in cui vale che $V_{MIN} = - V_{MAX}$. Possiamo esprimere questi 2 valori in termini di \textbf{Escursione Picco-Picco} $V_{pp}$:
\begin{gather}
    V_{pp} = V_{MAX} - V_{MIN}\\
    V_{MAX} = \frac{V_{pp}}{2}\\
    V_{MIN} = -\frac{V_{pp}}{2}\\
\end{gather}
Grazie a ciò, possiamo calcolarci la \textbf{potenza di picco} del segnale:
\begin{equation}
    P_{s,\text{picco}} = \left(\frac{V_{pp}}{2}\right)^2 = \frac{V^2_{pp}}{4}
\end{equation}
Sappiamo inoltre che per come abbiamo introdotto la quantizzazione, vale che:
\begin{equation*}
    \Delta = \frac{V_{pp}}{M} = \frac{V_{pp}}{2^n} \Rightarrow V_{pp} = 2^n\Delta
\end{equation*}
Allora possiamo scrivere il \textbf{Signal Noise Ratio di Picco}:
\begin{align*}
    SNR_{Q,\text{Peak}} &= \frac{\frac{V^2_{pp}}{4}}{\frac{\Delta^2}{12}} =\\
                        &= \frac{\frac{\Delta^2 2^{2n}}{4}}{\frac{\Delta^2}{12}}=\\
                        &= 3 \cdot 2^{2n}
\end{align*}
\paragraph{Potenza Media.}Oltre alla potenza di picco, possiamo esprimere la potenza del segnale in termini di \textbf{potenza media del segnale}:
\begin{equation}
    P_{s,\text{Media}} = f_{Picco} \cdot P_{s,\text{picco}}
\end{equation}
Dove $f_{Picco}$ è detto \textbf{fattore di picco}. Inoltre $0 \leq f_{Picco} \leq 1$.
Grazie a ciò possiamo esprimere il \textbf{Signal Noise Ratio Medio}:
\begin{align*}
    SNR_{Q,\text{Medio}} &= f_{Picco} \cdot \frac{P_{s,\text{picco}}}{P_Q} =\\
                         &= f_{Picco} \cdot SNR_{Q,\text{Peak}} =\\
                         &=  f_{Picco} \cdot 3 \cdot 2^{2n}
\end{align*}
Riconducendo dunque il rapporto di potenza segnale rumore in funzione dei \textbf{bit di campionamento} possiamo intuire che più
saranno i bit di campionamento, più il disturbo sarà minore, per questo motivo $n$ è anche detta \textbf{Risoluzione di Quantizzazione}. Inoltre, ogni bit in più di quantizzazione migliora il rapporto di un fattore 4.

\subsubsection{Decibel}
In generale, quando in fisica si ha a che fare con rapporto di potenze, ha più senso il confronto su scala logaritmica, per cui possiamo
definire i \textbf{decibel} sia in riferimento alle ampiezze, sia in riferimento alle potenze.
\begin{gather}
    dB_A = 20 \lg_{10} \frac{A_1}{A_2} \\
    dB_P = 10 \lg_{10} \frac{P_1}{P_2} 
\end{gather}
Le 2 formule sono state scelte in modo tale che il decibel per le ampiezze e per le potenze siano \textbf{invarianti} (parlare dell'una o dell'altra non fa differenza).

\paragraph{SNR in Decibel}
\begin{align*}
    dB(SBR_{Q,P}) &= 10 \lg_{10} \frac{P_{s,p}}{P_Q}=\\
                  &= 10 \lg_{10} 3 \cdot 2^{2n} =\\
                  &= 4,77 + 6,02n
\end{align*}
\begin{align*}
    dB(SBR_{Q,Media}) = dB(f_{picco}) + 4,77 + 6,02n
\end{align*}
Importante osservare che $dB(f_{picco}) < 0$ dal momento che $0 \leq f_{picco} \leq 1$ 
