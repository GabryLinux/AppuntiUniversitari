\section{Sistemi Lineari e Tempo-Invarianti (LTI) Discreti}
Consideriamo un sistema $S[\cdot] : y(n) = S[x(n)]$ con $n \in \mathbb{Z}$ Lineare e Tempo-Invariante; possiamo riscrivere $x(n)$
come somma infinita tra i prodotti tra l'impulso ideale e il loro peso, ossia come:
\begin{equation*}
    x(n) = \sum_{i = -\infty}^{+\infty} x(i)\delta(n-i)
\end{equation*}
Allora:
\begin{gather*}
    y(n) = S[x(n)] = \\ 
         = S\left[\sum_{i = -\infty}^{+\infty} x(i)\delta(n-i)\right] = \\
         =\sum_{i = -\infty}^{+\infty} x(i) \underbrace{S[\delta(n-i)]}_{h(n)}
\end{gather*}
Dal momento che $S[\cdot]$ è anche Tempo-Invariante, allora:
\begin{equation*}
    h(n - n_0) = S\left[\delta(n - n_0)\right]
\end{equation*}
\begin{highlightedeq}
    In definitiva, anche nel caso discreto, possiamo enunciare la proprietà fondamentale dei sistemi LTI, ossia che un qualsiasi sistema LTI
    può essere descritto come convoluzione tra l'ingresso e la risposta all'impulso del sistema:
    \begin{equation}
        y(n) = S[x(n)] = x(n) \ast h(n)
    \end{equation}
\end{highlightedeq}

\subsection{Causalità}
Un sistema $S[\cdot]$ LTI discreto è \textbf{Causale} se e solo se:
\begin{equation}
    h(n) = 0 \mbox{  } \forall n < 0, n \in \mathbb{Z}
\end{equation}
Dunque, come avevamo detto nel continuo,
\begin{equation}
    y(n) = x(n) \ast h(n) = \sum_{i = -\infty}^{n} x(i) h(n - i) = \sum_{i = 0}^{+\infty} h(i) x(n - i)
\end{equation}
Ossia un sistema LTI causale può essere scritto come la somma infinita del prodotto degli ingressi e degli impulsi dall'infinito passato al presente o come 
somma infinita degli ingressi presenti e passati per l'impulso

\subsection{Stabilità BIBO}
Per definizione, un sistema LTI $S[\cdot]$ è stabile BIBO se e solo se:
\begin{equation}
    \forall x(n) : |x(n)| \leq M < \infty \longrightarrow |y(n)| \leq N < \infty
\end{equation}
Ma possiamo ricavare un'altra definizione:
\begin{align*}
    |y(n)| &= \left|\sum_{i=-\infty}^{+\infty}h(i)x(n-i)\right|\\
    \left|\sum_{i=-\infty}^{+\infty}h(i)x(n-i)\right| &\leq \sum_{-\infty}^{+\infty} |h(i)||x(n - i)|\\
    \sum_{i=-\infty}^{+\infty} |h(i)||x(n - i)| &\leq M \sum_{-\infty}^{+\infty} |h(i)|
\end{align*}
Allora $S[\cdot]$ è stabile BIBO se $\sum_{i=-\infty}^{+\infty} |h(i)| < \infty$, cioè se la risposta all'impulso è assolutamente sommabile\\
\textbf{Ma è anche necessaria la condizione?}
Dimostriamo per assurdo
\paragraph{Ipotesi.}
\begin{enumerate}
    \item $y(n)$ è stabile
    \item $x(n)$ è limitata, $x(n) = \mbox{sign}[h(-n)]$
    \item $\sum_{i=-\infty}^{+\infty} |h(i)| = \infty$ 
\end{enumerate}
\paragraph{Dimostrazione.}
Calcoliamo $y(n)$ per $n = 0$:
\begin{align*}
    y(0) &= \sum_{i=-\infty}^{+\infty} |h(i)||x( - i)| =\\
     &= \sum_{i=-\infty}^{+\infty} |h(i)|\mbox{sign}[|h(i)|] =\\
     &= \sum_{i=-\infty}^{+\infty} |h(i)|
\end{align*}
Per ipotesi (3) $= \sum_{i=-\infty}^{+\infty} |h(i)| = +\infty$ dunque y(t) non è stabile, ASSURDO! Allora $= \sum_{i=-\infty}^{+\infty} |h(i)| < +\infty$ 
In definitiva possiamo affermare che:
\begin{equation}
    \mbox{Un sistema LTI è BIBO} \Longleftrightarrow  \sum_{i=-\infty}^{+\infty} |h(i)| < +\infty
\end{equation}

\subsection{Convoluzione Discreta}
Definiamo \textit{a(n) convoluto con b(n)} la seguente operazione 
\begin{equation}
    c(n) = a(n) \ast b(n) = \sum_{i = -\infty}^{+\infty} a(i)b(n - i)
\end{equation}

\subsubsection{Proprietà}
Enunciamo qui anche le proprietà di cui gode la convoluzione discreta:
\paragraph{Commutativa:} La convoluzione è commutativa, allora vale che
    \begin{equation*}
        f(n) \ast g(n) = g(n) \ast f(n)
    \end{equation*}
    \paragraph{Dimostrazione.}
    \begin{equation*}
        f(n) \ast g(n) = \sum_{i = -\infty}^{+\infty} f(i)g(n - i) 
    \end{equation*}
    Effettuiamo un cambio di variabile: 
    \begin{equation*}
        \begin{cases}
            m = n - i\\
            i = n - m
        \end{cases}
    \end{equation*}
    Riscriviamo:
    \begin{align*}
        \sum_{i = -\infty}^{+\infty} f(i)g(n - i) &= \sum_{n - m = -\infty}^{+\infty} f(n - m)g(m) = \\
                                                  &= \sum_{m = +\infty}^{-\infty} f(n - m)g(m)= \\
                                                  &= \sum_{m = -\infty}^{+\infty} g(m)f(n - m)= \\
                                                  &= g(m) \ast f(m)
    \end{align*}
    In questo caso possiamo scambiare senza problemi gli estremi della sommatoria dal momento che non dobbiamo preoccuparci del segno (mentre
    nell'integrale dovevamo considerare pure il $d\tau$ che dava un verso di integrazione).

\paragraph{Associativa} Ha la stessa formulazione della Convoluzione Continua \eqref{prop: convComm}
\paragraph{Distributiva (rispetto alla Somma)} La stessa formulazione della Convoluzione Continua \eqref{prop: ConvDistr}
\paragraph{Convoluzione con $\delta(n)$} La stessa formulazione della Convoluzione Continua \eqref{prop: ConvDelta}