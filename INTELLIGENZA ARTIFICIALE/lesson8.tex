\section{Adversarial Search}
Gli algoritmi di Search finora presentati hanno sempre tenuto conto di un ambiente Single-Agent in cui un obiettivo
si raggiungeva trovando un percorso in un grafo. Tutto cambia se invece consideriamo un ambiente Multi-Agent, in cui 
più agenti cercano di ottimizzare la propria utilità e ogni azione influenza tutti gli agenti. Tuttavia è sempre possibile
modellare un ambiente Multi-Agente, ossia un \textbf{gioco}, come un \textbf{Problema di Decisione}.\\
In particolare in un gioco:
\begin{itemize}
    \item Ogni agente deve ottimizzare una propria funzione di utilità
    \item Ogni azione influenza le azioni di tutti gli altri agenti
    \item Nella strategia da adottare bisogna tenere in considerazione anche le strategie adottate dagli avversari
\end{itemize}
Un'altra differenza che possiamo riscontrare con i problemi di Search è che, alla fine del gioco, non si raggiunge un goal
ma si vince una \textit{ricompensa}, o \textit{payoff}, e ciò dipende dal risultato del gioco.

\subsection{Caratteristiche di un gioco}
In teoria dei giochi, un gioco può essere caratterizzato da:
\begin{itemize}
    \item \textbf{Numero di giocatori}: a 2 (caso più semplice) oppure a $n$ giocatori (caso più difficile)
    \item \textbf{Tipi di Agenti}: \textit{razionali} (cioè che in ogni occasione scelgono sempre di ottimizzare la propria utilità) o \textit{$\epsilon-$razionali} (ossia che scelgono di trovare una soluzione peggiore al massimo $\epsilon$)
    \item \textbf{Struttura Sequenziale}: a \textit{turni} o \textit{ad azioni sequenziali} (in realtà è stato dimostrato che una struttura a turni è una generalizzazione della struttura ad azioni sequenziali)
    \item \textbf{Esito delle azioni}: \textit{deterministico} o \textit{stocastico}
    \item \textbf{Struttura dei Payoff}: \textit{a Somma Costante} oppure \textit{a Somma Generica}
    \item \textbf{Ad Informazione}: \textit{completa}, se si conoscono tutti gli obiettivi e azioni degli altri agenti, altrimenti \textit{incompleta}
    \item \textbf{Memoria del Gioco}: \textit{perfetta} se ogni agente conosce TUTTE le mosse compiute dagli altri agenti anche in precedenza, altrimenti \textit{imperfetta}
\end{itemize}
Dal momento che i giochi possono essere molto complessi e le interazioni tra gli agenti non facili da studiare, sono nati 2 rami di teoria dei giochi che studiano i comportamenti
degli agenti in modo diverso:
\begin{itemize}
    \item \textbf{Teoria dei Giochi Competitiva}: studia i comportamenti degli agenti individualmente, come se ogni agente considerasse solo il proprio tornaconto
    \item \textbf{Teoria dei Giochi Cooperativa}: studia le dinamiche delle formazioni delle coalizioni.
\end{itemize}

\begin{tcolorbox}[documentation note, title={Nota Importante}]
   In questo corso considereremo giochi: a 2 giocatori, ad agenti razionali, a turni, deterministico, ad informazione perfetta e a somma zero (ossia costante).
\end{tcolorbox}

\subsubsection{Formalizzazione di un Gioco}
Descrivo qui la notazione usata per descrivere i giochi e i relativi algoritmi all'interno di questo corso:
\begin{itemize}
    \item \textbf{Insieme di Stati}: $S = \{s_1,s_2, \dots, s_n\}$
    \item \textbf{Stato Iniziale}: $s_k \in S$
    \item \textbf{Agenti}: $I = \{i_1, i_2\}$
    \item \textbf{Azioni Possibili}: $A=\{a_1, a_2, \dots, a_n\}$
    \item \textbf{Turno}: è l'insieme delle azioni possibili $A(s_k)$ di un giocatore $I(s_k)$ con $s_k \in S$
    \item \textbf{Modello di Transizione}: $f(s_k, a)$ con $a = A(I(s_k))$ l'azione effettuata in un turno e $s_k$ lo stato precedente
    \item \textbf{Stato Terminale}: $s_t \in S$ è stato terminale se e solo se verifica un predicato $T$ che determina la fine di un gioco, ossia se $T(s_t) = 1$
    \item \textbf{Payoff}: $u_i(s_t)$ è il payoff del giocatore $i$ nello stato terminale $s_t$
\end{itemize}

\subsubsection{Gioco A Somma Zero}
Per presentare il primo algoritmo di risoluzione di un gioco , prenderemo in considerazione un gioco a somma zero:\\
Dato un gioco a 2 agenti, tale si definisce \textit{a somma zero} se e solo se:
\begin{equation}
    u_i(s_t) + u_2(s_t) = 0 \mbox{    } \forall s_t \in S
\end{equation}
Dove $s_t$ è un qualunque stato terminale.
Considerare un gioco a 2 agenti a somma zero permette di semplificare molto i calcoli dal momento che è possibile riscrivere l'utilità di un agente in funzione dell'utilità dell'altro agente.
In particolare, massimizzare il proprio payoff significa minimizzare il payoff avversario dal momento che l'avversario perde tanto quanto vinco, ossia:
\begin{equation}
    \max\{u_2\} = \max\{-u_1\} = -\min\{u_1\}
\end{equation}

\subsubsection{Strategie}
Quello che abbiamo finora descritto sono le \textit{regole} del gioco; un'altra importante parte dei giochi riguarda le \textit{strategie} adottate dagli agenti per 
massimizzare il payoff. Le strategie si dividono in 2 grandi categorie:
\begin{itemize}
    \item \textbf{Strategia Pura}: se la strategia $\sigma_i$, del giocatore $i$, sceglie, secondo un criterio, un'azione possibile in uno stato del gioco appartenente a $A(s_k)$, ossia:
    \begin{equation}
        \sigma_i : \{s_k \in S | I(s_k) = i\} \longrightarrow A(s_k)
    \end{equation}
    \item \textbf{Strategia Mista}: se la strategia $\sigma_i$, del giocatore $i$, sceglie un'azione possibile in uno stato del gioco appartenente a $A(s_k)$ secondo uno
    spazio di probabilità $\Pi(A(s_k))$, ossia:
    \begin{equation}
        \sigma_i : \{s_k \in S | I(s_k) = i\} \longrightarrow \Pi(A(s_k))
    \end{equation}
\end{itemize}

\subsubsection{Albero di Gioco}
Sebbene un problema di ricerca e un gioco siano sostanzialmente differenti, è ancora possibile modellare un gioco a mo' di albero, detto appunto albero di gioco, 
ed applicare un algoritmo di ricerca tra quelli presentati precedentemente per trovare la strategia ottima dell'agente. In questo albero, ogni nodo è detto \textit{nodo
di decisione} e i nodi foglia sono il nodo terminale del gioco con annesso payoff ai vari giocatori. Una strategia è un percorso all'interno dell'albero, ossia una sequenza di mosse
che porta ad uno stato terminale.

\subsection{Minimax}
Minimax è un algoritmo \textbf{Corretto e Completo} che trova sempre la strategia ottima per massimizzare i payoff di un agente e lo fa esplorando l'albero di gioco.
Lavora con l'assunzione che l'agente avversario sia \textbf{razionale}.
Quest'algoritmo:
\begin{itemize}
    \item è basato su DFS \eqref{alg: dfs}
    \item è basato su Backtracking
    \item effettua il \textbf{backup/riporto} dei valori al nodo padre 
\end{itemize}
\subsubsection{Descrizione Gioco}
Il gioco risolto da questa prima implementazione di Minimax è la seguente:
\begin{itemize}
    \item \textbf{Agenti} $I = \{\mbox{MAX},\mbox{MIN}\}$
    \item \textbf{Gioco a Somma Zero}
    \item Max avrà sempre utilità non negativa e le conseguenze di MIN saranno sempre negative
\end{itemize}

\subsubsection{Descrizione Algoritmo}
\begin{enumerate}
    \item Si effettua una DFS sull'albero di gioco
    \item Ogni volta che si esplora totalmente un sottoalbero:
        \begin{itemize}
            \item se il nodo padre di tale sottoalbero è di \textbf{MAX}, allora riporto al padre il valore \textbf{maggiore} tra quelli riportati dai figli
            \item se il nodo padre di tale sottoalbero è di \textbf{MIN}, allora riporto al padre il valore \textbf{minore} tra quelli riportati dai figli
        \end{itemize}
    \item Arrivati al nodo radice si trova il \textit{Valore di Gioco}.
    \item Si trova la strategia (ossia il percorso) che genera al payoff uguale al valore di gioco.
\end{enumerate}
Bisogna tenere a mente che tale algoritmo lavora sempre con l'assunzione che l'agente avversario sia razionale, perchè se non lo è,
l'algoritmo non garantisce di trovare il payoff ottimo; tuttavia ci dà una garanzia: il payoff sarà SEMPRE almeno il valore di gioco.

\subsubsection{Minimax a $n$ agenti}
La strategia adottata da Minimax può essere estesa a più agenti, tuttavia dovremmo fare alcuni cambiamenti: non ha più senso trattare 
giochi a somma zero (poichè a più giocatori non ci dà alcun vantaggio nei calcoli); inoltre il payoff di un gioco non può essere
più espresso da uno scalare ma da un vettore di payoff, detto profilo del payoff, di dimensione pari al numero di giocatori.