\section{Markov Decision Processes}
Tutti i problemi finora affrontati ricadevano sempre entro la famiglia dell\textbf{inferenza}. D'ora in poi tratteremo l'altro
ramo dell'IA, ossia l'\textbf{apprendimento}.
La prima classe di problemi che affronteremo in questo ambito sono i \textbf{Markov Decision Processes}.
\paragraph{Descrizione Generale.}
\begin{itemize}
    \item Gli agenti hanno come obiettivo il raggiungimento di uno \textbf{stato terminale} (può essere unico, multiplo o assente se l'agente dev'essere sempre attivo)
    \item L'ambiente è \textbf{Stocastico} e \textbf{Single-Agent}
    \item Esiste un \textbf{Orizzonte Temporale} entro cui agire
\end{itemize}
\paragraph{Formalizzazioni.}
\begin{itemize}
    \item \textbf{Insieme degli Stati}: $S : \{s_1, s_2, \dots\}$
    \item \textbf{Stato Iniziale}: $s_i \in S$
    \item \textbf{Stato Terminale}: $s_T \in S$
    \item \textbf{Azioni possibili per lo stato s}: $A(s)$
    \item \textbf{Modello di Transizione Stocastica}: dato $s \in S, a \in A(s), s' \in S$ con la transizione da $s$ a $s'$ con azione $a$,
    il modello di transizione $f$ indica la probabilità di finire proprio sullo stato s', ossia:
    \begin{equation}
        f(s,a,s') =  \mathbb{P}(s' | s,a)
    \end{equation} 
    \item \textbf{Reward}: ad ogni transizione l'agente riceve un reward (il corrispondente di costo negli algoritmi di search) che 
    dev'essere \textbf{additivo}
    \item \textbf{Orizzone H}: ossia le epoche di decisione per un MDP; oltre quest'epoca, tutto non viene più considerato. Se $H = \infty$ allora
    il processo termina se e solo se l'agente raggiunge uno stato terminale
\end{itemize}
\subsection{Markovianità}
Dato un processo stocastico che genera una \textbf{sequenza aleatoria} $s_1, s_2, \dots, s_n$ (aleatoria perchè l'ambiente è stocastico),
un processo si dice \textbf{Markoviano} se e solo se \textit{l'esito di un'azione dipende solo dallo stato corrente e non dagli stati precedenti}, ossia:
\begin{equation}
    \mathbb{P}(s_{t+1} | s_t,a_t,s_{t-1},a_{t-1},\dots, s_0, a_0) = \mathbb{P}(s_{t+1} | s_t,a_t)
\end{equation}
In parole povere, se un processo è markoviano, tutte le informazioni necessarie per potere scegliere la mossa migliore è già contenuta nello stato attuale stesso.
Assumere che un processo sia markoviano, inoltre permette di facilitare di molto i calcoli non perdendo di generalità (questo perchè la markovianità è un'assunzione realistica in molti contesti).

\subsection{Risoluzione di un MDP}
In un problema MDP, quello che si cerca è la \textbf{policy} $\pi$:
\begin{equation*}
    \pi : S \rightarrow A
\end{equation*}
Questa è una funzione \textbf{deterministica} che mappa ad ogni stato un'azione. Il nostro obiettivo è, una volta risolto il problema,
quello di rendere l'agente un \textbf{agente reflex} che sa già, ad ogni stato, le azioni da compiere. Naturalmente la difficoltà sta nel 
trovare una policy che \textbf{ottimizzi} un parametro (ad esempio il reward).

\subsection{Stimare la bontà di una policy}
Rispetto ai problemi precedenti, stimare la bontà di una strategia o un percorso risultava essere piuttosto intuitiva. Qui le cose si complicano 
per 2 motivi:
\begin{itemize}
    \item Le transizioni sono stocastiche, dunque i reward sono \textbf{incerti}
    \item Viene imposto un \textbf{orizzonte temporale }$h$
\end{itemize}
La \textbf{policy ottima} deve dunque \textbf{massimizzare il valore atteso della somma dei reward}.
\paragraph{Definizione}
\begin{itemize}
    \item Data una sequenza $s_1,\dots, s_n$ di stati generati da $\pi$
    \item Dati un reward associato ad ogni stato $R_1,R_2,\dots, R_n$
    \item Data $U(s_1,s_2,\dots,s_n) = R_1 + R_2 + \dots + R_n$
\end{itemize} 
La policy ottima $\pi^*$  deve generare il valore atteso della somma dei reward non peggiore di ogni altra policy:
\begin{equation}
    \pi^* = \arg \max_{\pi} \mathbb{E}_{\pi}\left[ U(s_0,s_1, \dots, s_n)\right]
\end{equation}
Possiamo dunque notare come i reward assegnati influenzino notevolmente la policy:
\begin{itemize}
    \item per $R < 0$, possiamo notare che la policy ottima induca l'agente a trovare lo stato terminale migliore nella maniera più efficiente
    \item per $R \ll 0$ possiamo notare che la policy ottima induca l'agente a trovare un qualsiasi stato terminale nel minor numero di passi possibili
    \item per $R > 0$ la policy ottima indurrà l'agente a non raggiungere mai lo stato terminale  
\end{itemize}