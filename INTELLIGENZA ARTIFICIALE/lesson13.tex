\subsection{Reward Scontati}
Ponendo all'infinito l'orizzonte, l'unico modo che ha l'agente di terminare è quello di finire su uno stato terminale; dunque
se la policy lo porta ad uno stato terminale, il valore atteso della somma dei reward è finita. Tuttavia, nel caso in cui ciò
non accadesse, il valore atteso della somma dei reward risulta essere infinita (perchè transita in infiniti stati con altrettanti infiniti reward).
Per cui dobbiamo introdurre una nuova definizione di somma di reward, ossia la \textbf{somma di reward scontati}:
\begin{equation}
    U(s_0,s_1,\dots) = \gamma^0R_0 + \gamma^1R_1 + \dots = \sum_{t = 0}^{H}\gamma^tR_t
\end{equation}
Dove $0\leq\gamma\leq 1$ e viene detto \textbf{fattore di sconto}. La combinazione sopra descritta ha la forma di un 
\textbf{decadimento esponenziale} e scegliendo un apposito valore di $\gamma$ possiamo bilanciare l'importanza tra l'immediato
futuro e il futuro più remoto:
\begin{itemize}
    \item per $\gamma \approx 1$ la somma dei reward si comporta esattamente come combinazione lineare con peso unitario dei reward che avevamo introdotto precedentemente, e dunque vengono preferite le scelte più vantaggiose nel futuro più lontano.
    \item per $\gamma \approx 0$ la somma dei reward tiene in considerazione i termini più vicini temporalmente, per cui sono più importanti le scelte che influenzano l'immediato futuro.
\end{itemize}
Si può benissimo dimostrare che per $H = +\infty$ la serie converge:
\begin{equation}
    \sum_{t = 0}^{+\infty} y^tR_t = \frac{R_\text{MAX}}{1 - \gamma}
\end{equation}

\subsection{Policy Stazionaria}
Paradossalmente, considerare problemi in cui l'orizzonte è finito, rende la risoluzione del problema molto complessa:
questo perchè è necessario considerare non solo i reward per stimare la bontà di una policy, ma anche il budget di mosse rimaste.
Dal momento che il \textbf{passato} conta nelle stime delle policy in problemi a orizzonti finiti, la \textbf{Markovianità si perde} poichè
la policy non è più \textbf{stazionaria}, ossia, non considera solo il reward del nodo in un tempo qualsiasi. Proprio per tutti questi motivi, gli MDP che affronteremo hanno:
\begin{itemize}
    \item $H = \infty$
    \item Somma di reward scontata
\end{itemize}

\subsection{Ricerca Policy Ottima}
Arrivati a questo punto, dobbiamo calcolare in qualche modo $\pi^* : S \rightarrow A$ che masasimizza il \textbf{valore atteso
della somma di reward scontati}; analizziamo ora 2 principali approcci:
\paragraph{Programmazione Dinamica.}La Programmazione Dinamica è un approccio risolutivo che consiste nel suddividere un problema
P in tanti sottoproblemi $P_1,P_2,\dots$ e risolverli separatamente. Dalle sottosoluzioni si ricava la soluzione al problema P di partenza.
I problemi che possono essere risolti tramite prog. Dinamica devono rispettare il \textbf{principio di ottimalità di Bellman} ossia:
\begin{quotation}
    Nella soluzione ottima di P sono contenute le soluzioni dei sottoproblemi $P_1,P_2,\dots$ e viceversa
\end{quotation}
Non tutti i problemi rispettano questo principio, es:
\begin{itemize}
    \item Cammino Minimo su Grafo (rispetta)
    \item Cammino Massimo su Grafo pesato (non rispetta)
\end{itemize}
Si può dimostrare (per fortuna) che MDP rispetta il principio di Bellman, allora esiste una procedura risolutiva realizzabile tramite Programmazione Dinamica.
Per fare ciò dobbiamo introdurre 2 nuove metriche per determinare la bontà di $\pi^*$:
\begin{itemize}
    \item La \textbf{Value function} $V^*(s)$ di uno stato $s$: indica il valore atteso della somma scontata dei reward da s in avanti
    seguendo la policy ottima $\pi^*$. Possiamo osservare che $V^*(s)$ è indipendente dal tempo dal momento che $\pi^*$ è stazionaria.
    \item L'\textbf{Action value function} $Q^*(s,a)$: misura il reward che ottengo eseguendo $a$ (che non necessariamente dev'essere
    prescritta dalla policy ottima) e ipotizzando che dal nodo successivo in poi seguo la policy ottima
\end{itemize}
Esiste una relazione tra le 2 funzioni:
\begin{equation}
    V^*(s) = \max_a Q^*(s,a)
\end{equation}
Conoscere $V^*(s)$ significa dunque conoscere la policy ottima, dunque significa risolvere il problema.\\
Dalla definizione di $Q^*$ possiamo dedurre un'altra relazione tra la value function e l'action function, ossia:
\begin{equation}
    Q^*(s,a) = \gamma R(s) + V^*(s')
\end{equation}
Ora, diamo una definizione più precisa di $Q^*$:
\begin{equation}
    Q^*(s, a) = \sum_{s'} 
\underbrace{\mathbb{P}(s' \mid s, a)}_{\substack{\text{è la probabilità} \\ \text{di transitare su } s'}} 
[ 
\underbrace{R(s, a, s')}_{\substack{\text{è il reward} \\ \text{immediato}}} 
+ 
\underbrace{\gamma V^*(s')}_{\substack{\text{è lo sconto} \\ \text{del reward successivo}}} 
]
\end{equation}
Da tutte queste equazioni possiamo ottenere l'equazione che le riassume tutte:
\begin{equation}
    V^*(s) = \max_a \sum_{s'}\mathbb{P}(s'|s,a)[R(s,a,s') + \gamma V^*(s')]
\end{equation}
E viene detta \textbf{Equazione di Bellman}.
Possiamo vedere $V^*(s')$ come il sottoproblema da risolvere mentre il sottoproblema terminale è lo stato terminale e ha valore pari a 0.
Un'altra considerazione che possiamo fare è che possiamo mettere a sistema tutte le $V^*(s)$ per ogni s ed ottenere un sistema.
SI può dimostrare che la soluzione a questo sistema è automaticamente la soluzione ottima, dal momento che l'equazione di bellman ammette
un'unica soluzione.